\section{Related Work}

The previous studies demonstrate that infants have a mechanism to process
statistical properties of natural language. \cite{saffran1996statistical} 
states that 8-month-old infants have a mechanism to process
statistical properties of natural language. Adults also
are sensitive to co-occurrence patterns beyond bigram. In line
with this study, \cite{hahn2012measuring} shows that conditional probabilities
calculated with 4-word contexts are correlated with cloze probabilities 
which is a measure of relatedness of a word to a sentences calculated
with EEG signals of the participants. \cite{arnon2010more},\cite{romberg2010statistical} 
provides a review of statistical mechanisms for language acquisition.

Distributional hypothesis or knowledge is a statistical approach to natural
language. The distributional hypothesis suggests that words 
occurring in the similar contexts tend to have similar meaning 
and grammatical properties \cite{harris1954word}. 
\cite{gomez2002variability} and \cite{van2010linking} demonstrate
that distributional statistics help infants acquire non-adjacent
dependencies. \cite{monaghan2012integrating} proposes that
exploiting distributional regularities of function words is helpful in acquiring word-referent
mappings. Distributional knowledge is also useful in word segmentation task \cite{saffran1996word}.
\cite{thiessen2012iminerva} proposes that an unified model exploiting distributional
statistics may be capable of handling variety of language tasks. Their experiments
demonstrate that a memory-based distributional framework is 
successful in the tasks of phonetic discrimination, a word learning , and non-adjacent
association.

The use of distributional knowledge for syntactic category acquisition
is well studied. The two success criteria for syntactic category
acquisition are accuracy and completeness. Accuracy measures how
accurate the predictions were at grouping the words into the
same grammatical categoty together. It is defined
as the total number of correct category predictions
over total number of predictions. The completeness, on the other hand
measures how well a given category is predicted. The completeness
is equal to number of correct predictions for a category divided
by number of correct predictions summed with number misses in that category.
\cite{Redington98distributionalinformation} defines the context of
a word as the previous and following words. With this definition, they construct
context vectors of target words for clustering. Using average link clustering
with a threshold maximizing accuracy and completeness, target  words are
separated into categories. Although the categorizations are generally accurate,
the method lacks of completeness. In addition, the underlying process to determine
the threshold for infants is not clear \cite{ambridge2011child}. \cite{cartwright1997syntactic} introduces an incremental learning framework for syntactic category acquisition. They claim that no language 
learner is exposed to all the sentences of the language to learn
syntactic categorization. Therefore, their framework, exploiting
distributional knowledge to confine the search space, forgets the sentence
after processing it. To accomplish this, they convert the syntactic category acquisition into 
Minimum Description Length optimization problem. Their results are 
similar to \cite{Redington98distributionalinformation}, high in 
accuracy but low in completeness. Still, it shows that distributional
knowledge is powerful for syntactic categorization even if the learner is
exposed the small portion of the syntactic input. \cite{Mintz200391} proposes frequent frames. A frequent frame consists of two jointly appearing words with one word in the middle and co-occur
frequently. Experiments on child directed speech reveal that 
frequent frames have the ability to assign word categories with high
accuracy. Though the accuracy is high, the same as the previous work, it suffers from
completeness. As \cite{clair2010} points out, frequent frames suffer from coverage.
\cite{clair2010} combines the bigram's 
coverage power \cite{Redington98distributionalinformation} and \cite{monaghan2008integration}
 and accuracy of frequent frames \cite{Mintz200391}. The 
experiments demonstrate that infants make use of both bigram and 
trigram sources. As they pointed out, they may even use higher-order 
relationships between words. 

\cite{freudenthal2005resolution} points out a complication of distributional
methods for constructing syntactic categories. Distributional methods suggest 
that words occurring in the similar context  can be used interchangebly. They 
claim the evaluation methods used in studies like \cite{Redington98distributionalinformation, monaghan2008integration} or \cite{Mintz200391} could be misleading. Specifically, if a word is substituted with another one in its
category, the resulting sentences could be erroneous in a way that they are not observed in
infants' speech. As a success criteria, they argue that the proposed categorization
should generate plausible sentences. They introduce chunking mechanism merging 
words that are seen frequently. The mechanism seems
successful to generate meaningful sentences, still, the
proposed solution is computationally complex to disclose the learning mechanism
in infants.

More recently, \cite{alishahi2012concurrent} proposes an incremental learning
scheme inducing soft word categories while learning the meaning of words. 
\cite{thothathiri2012effect}
examines the role of prosody on infants' distributional learning of
syntactic categories and concludes that the prosody shows little
influence. \cite{reeder2013shared} aims to answer the use of distributional knowledge
when the evidence on  the possible context of a word is not enough. Furthermore,
they explain how and when the language users form new categories depending on 
the overlaps between the context words. They claim that generalization
or restriction of category rules is done in probabilistic manner.
