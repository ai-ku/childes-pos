\appendix
\section{Preprocessing}
\label{app:preprocessing}
We apply following preprocessing steps initial to our analysis:
\begin{itemize}
\item All punctuation, pause, trailing off and interruption marks are
  treated as utterance boundaries.
\item Repetitions of a word are kept in the text and their grammatical
  categories are automatically set to the grammatical category of the
  original word.
\item Words that are grammatically necessary but not spoken are
  deleted (grammatical omissions).
\item Shortenings, dropping sounds out of words, are ignored and converted to
  the corresponding actual word forms.
\item Untranscribed words such as {\it xxx} or {\it yyy} are removed. 
\item Assimilations, complex sound changes of words or word phrases, are not
  converted to the actual form.
\end{itemize}

\section{Frequent Frames}
\label{app:mintz03}
In this section we examine the corpus statistics of the top 45 fixed frames
($aXb$) and replicate \cite{Mintz200391} in order to show how informative
frequent frames are on grammatical categorization compared to the frequent
bi-gram frames($aX$ and $Xb$) \cite{clair2010}.

\section{Experiment \arabic{ExperimentCounter}: Corpus analysis}
\begin{table}[ht]
  \small
  \centering
  \caption{10-fold cross-validation classification accuracies of models based
    on flexible frames ($aX + Xb$) and substitutes ($a*b$) on each child corpus
    after 100K training patterns are summarized.  Standard errors are reported
    in parentheses.  Lambdas of $aX+Xb$ and $a*b$ are both tested against each
    other and zero association by using two tailed z-test.  All tests have
    $p<.001$.} \begin{tabular}{lccccc}
    \hline
    Corpus & \multicolumn{2}{c}{$aXb$} 
    && \multicolumn{2}{c}{$aX+Xb$}\\ 
    \cline{2-3}
    \cline{5-6}
    & Accuracy & $\lambda$ && Accuracy & $\lambda$\\
   \hline
    Anne  & .5416 (.0224)& .3099 (.0255) &&.7628 (.0075) & .6407 (.0124)\\
    Aran  & .5156 (.0215)& .2837 (.0120) &&.7337 (.0059) & .5977 (.0081)\\
    Eve   & .5370 (.0258)& .3209 (.0130) &&.7580 (.0068) & .6351 (.0083)\\
    Naomi & .5229 (.0244)& .2840 (.0220) &&.7316 (.0086) & .5892 (.0113)\\
    Nina  & .5636 (.0113)& .3287 (.0183) &&.7755 (.0040) & .6547 (.0075)\\
    Peter & .5661 (.0180)& .3541 (.0206) &&.7670 (.0071) & .6518 (.0088)\\
    \hline
  \end{tabular}
  \label{t:frame100K}
\end{table}




