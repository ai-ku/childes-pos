\section{General Discussion}

This study proposes paradigmatic representations of context
as opposed to syntagmatic representations for syntactic category
acquisition. The paradigmatic approach
suggests to use probable substitutes of word ($a*b$). On the other
hand the syntagmatic approach proposes to use the preceding bigram and the
succeeding bigram whichever is fruitful ($aX + Xb$).

In order to contrast these two representations we replicate the 
experimental setup of \{clair2010}. Experiments show that
when the models exposed to limited amount of training patters
the $a*b$ is significantly more accurate than $aX + Xb$. Results
of long training phase show the same pattern, however, the gap
between these approaches gets smaller.

We investigate the dependency of the model to the number of substitutes. In
this experimental setup the number of substitutes varies from 1 to 64. The results
show that the accuracy of the model dramatically increases up to 16. After
16 substitutes, no significant improvement in accuracy is observed. We
conclude that the model is robust as long as 16 substitutes are observed.

We explore the effect of the n-gram order of language model to the
accuracy of the model. While determining the probability of the next word 
in a sequence of words, n-gram order determines how many preceding should be considered.
We hypothesise that order of n-gram determines how accurate the substitutes
of a target word. Thus, it should affect the ($a*b$) model's accuracy. Figure~\ref{fig:perplexity}
and Figure~\ref{fig:perplexity} show that the model's performance 
highly dependend on the n-gram order of the language model.
