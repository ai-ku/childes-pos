\section{Introduction}
\label{sec:introduction}
\subsection{Psycholinguistic evidence relevant to substitutes}

\subsection{Comparison with previous distributional approaches}

Following two examples\footnote{These examples are extracted from the Anne
corpus and substitute probabilities are calculated as described in
Section~\ref{sec:substitute_vectors}.} illustrate the advantage of paradigmatic
representations in uncovering similarities where no overt similarity that can
be captured by a syntagmatic representation exists. The word ``you'' from the
first sentence and the word ``I'' from the second sentence have no common
neighbors.  The paradigmatic representation captures the similarity of these
words by suggesting the same top substitutes for both (the numbers in
parentheses give substitute probabilities): 

\begin{quote}
  \small
  % Anne corpus sentence id:13228
  (1) \noindent{\em ``they fall out when {\bf you} put it in the box .''}\\
  \noindent{\bf you:} you(.8188), I(.1027), they(.0408), we(.0146) $\ldots$
\end{quote}

\begin{quote}
  \small
  % Anne corpus sentence id: 13085
  (2) \noindent {\em ``what have {\bf I} got here ?''}\\
  \noindent {\bf I:} we(.8074), you(.1213), I(.0638), they(.0073) $\ldots$
\end{quote}

The high probability substitutes reflect both semantic and syntactic properties
of the context.  Top substitutes for ``I'' and ``you'' are not only pro-nouns, but
specifically pro-nouns compatible with the semantic context.  Top substitutes for
the word ``fall'' in the first example consist of words that are also  verbs:
come(.7875), go(.0305), fall(.0232), were(.0187) $\ldots$.


