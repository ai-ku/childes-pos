\begin{enumerate}
\item We need to clarify the tag set that is used during the
  experiments. May be it is better to give the whole mapping as an
  Appendix section.
\item Data statistics (it is common to all experiments)
\end{enumerate}
\section{Experiment 1: corpus analysis}
In this section we replicate the corpus analyses of \cite{clair2010}
and \cite{Mintz200391}.  

\subsection{Input Corpora}

In order to be consistent with \cite{clair2010} and \cite{Mintz200391},
we use the same six corpora of child-directed speech from the CHILDES
corpus \citep*{macwhinney2000childes}: Anne and Aran
\citep*{theakston2001role}, Eve \citep*{JCL:1765112}, Naomi
\citep*{sachs1983talking}, Nina \citep*{suppes1974semantics}, Peter
\citep*{Bloom1974380, bloom1975structure}.  Following
\cite{Mintz200391} we only analyze the adult utterances in sessions
where the target child is 2.6 years old or younger.

\subsubsection{Preprocessing}

The grammatical category of words in CHILDES are extracted by first
applying the MOR parser \citep*{macwhinney2000childes} and then using
the POST disambiguator \citep*{sagae2004automatic}.  The accuracy of
CHILDES grammatical categories is approximately 95\%
\citep*{parisse2000automatic} and it is encoded in the MOR line of the
CHILDES corpus.

We apply the following pre-processing steps \citep*{clair2010} initial
to our analyses:
\begin{itemize}
\item All punctuation, pause, trailing off and interruption marks are
  treated as utterance boundary marks.
\item Repetitions of a word are kept in the text and their grammatical
  categories are automatically set to the grammatical category of the
  original word.
\item Words that are grammatically necessary but not spoken are
  deleted (grammatical omissions).
\item {\bf Short usages ??}
\item {\bf Frames do not include utterance
    boundaries. \cite{Mintz200391}}
\end{itemize}

\subsection{Method}
%% How do we extract frames? Which words are the our target words
Word sequences that consist of three words and do not contain any
utterance boundaries are extracted for each child corpus seperately.
The first and third words of the word sequences are treated as frame
elements while the middle word is the target word that we want to
categorize.  The correct grammatical category of target words are
extracted from CHILDES.

%% Define frames in here again? also in related work.
%% What is standard labeling? 
%% Are we going to report accuracy and completeness

%% Do we replicate the 45 tag case?  No.  \cite{}
\subsection{Results}

\section{Experiment 2: computational modeling of substitutes}

% Why do we choose feedforward connectionist model?
\cite{clair2010} uses feed-forward connectionist model to compare the
effect of distributional cues from various frame types on the
grammatical category problem.  We adopt their framework to compare the
paradigmatic representation (substitute words) with the syntagmatic
representation (flexible frames).
% What are the input and output layer
% What does the connectionist model do? Briefly explain without giving
% too much mathematical details
% Two aspects:  
% description of learning process
% how to represent distributional

A prototypical connectionist model consists of input, hidden and
output layers.  Input and output layers are connected to each other
through the hidden layer.  The behavior of the output units are
determined by the activity of the hidden layers which is triggered by
the input layer.

\subsection{Method}

We train two connectionist models to compare flexible frames ($aX+Xb$)
to the substitute words ($a*b$).

\subsection{Architecture}
% How do they represent the frame information?
For each model we input the distributional information to the
feed-forward connectionist model in the following way,

\begin{itemize}
%#\item $aXb$: Each input unit represents a distinct frame thus only one
% unit is activated (i.e. set to 1) for each target word.
\item {\bf$aX+Xb$ model:} The first and second half of the input units
  correspond to the preceding bigram ($a$) and the succeeding bigram
  ($b$), respectively.  Thus two input units are activated for each
  target word.
\item {\bf $a*b$ model:} Each input unit represents a distinct
  substitute and input units that correspond to the substitutes of the
  target word are set to the number of their occurrences in the
  sampled set.
\end{itemize}

% Give an example sentence to show how we represent each frame
Each output unit represents a distinct grammatical category therefore
the models are expected to produce only one active (non-zero) output
unit for each target word.  If there are more than one active units
present in the output layer\footnote{why neural network produces more
  than one active unit.}, the target word is assigned to the
corresponding grammatical category of the largest unit.

Both models have 10 output units due to the standard labeling
\citep*{Mintz200391}.

Number of hidden units is set to 200 and initialized randomly for each
model. {\bf backprobagation(0.1), learning rate, sigmoid...}

\subsection{Training and Testing}

%%% How do we split the train and test
% 10-fold cross valdiation
% -> define/implementaion
% -> advantage

We analyze each child corpus seperately and apply 10-fold cross
validation to measure model performances.  10-fold cross validation
randomly splits the sentences of each corpora into 10 sentence wise
equal sized subsamples.  Thus target words from the same sentence are
observed in the same subsample.  A single subsample is kept as the
test data while remaining 9 samples are used as the training data.
The process is repeated until all subsamples are used exactly once as
the test data and report the average accuracy of the 10 runs.  The
main advantage of cross validation is that each subsample is observed
both as test and training set.

To compare the effects of paradigmatic representation with the
syntagmatic one we test both models on each child corpus using the
same 10-fold cross validation split.

\begin{table}[ht]
\centering
\caption{10-fold cross-validation classification accuracies of flexible frame
  ($aX + Xb$) and paradigmatic models ($a*b$) on each child corpus after 100K
  training patterns are summarized.  Standard errors are reported in
  parentheses.
}
\begin{tabular}{|l|l|l|}
  \hline
  Child & $aX + Xb$ & $Substitutes$\\
  \hline
  Anne  & .7833 (.0072) & {\bf .8335 (.0078)} \\
  Aran  & .7495 (.0091) & {\bf .8218 (.0063)} \\
  Eve   & .7896 (.0073) & {\bf .8467 (.0113)} \\
  Naomi & .7643 (.0122) & {\bf .8212 (.0132)} \\
  Nina  & .8053 (.0083) & {\bf .8555 (.0058)} \\
  Peter & .7924 (.0083) & {\bf .8509 (.0083)} \\
  \hline
\end{tabular}
\label{t:framevssub}
\end{table}

% \begin{table}[ht]
% \small
% \centering
% \caption{The 70\% of the sentences in the union of 6 child corpora are
%   used as the training set while the remaining 30\% sentences of each
%   corpus are used as the test set.  Average classification accuracy
%   (10 runs) of the supervised connectionist model for the standard
%   labelling on each child corpus after 10K words of training are
%   summarized and the corresponding standard errors are reported in
%   parentheses.}
% \begin{tabular}{|l|l@{ }|l@{ }|l@{ }|l@{ }|l@{ }|l@{ }|l@{ }|}
%   \hline
%   & \multicolumn{4}{c|}{Frames} & \multicolumn{2}{c|}{\specialcell{Number of \\ Substitute Words}}\\
%   \hline
%   Child & aX & Xb & aXb & aX + Xb & 1 & 16\\
%   \hline
%   Anne & .5317 (.0374)  & .5065 (.0214) & .3788 (.0279) & .6190 (.0253) & .6221 (.0333) & .7829 (.0059) \\
%   Aran & .5098 (.0344) & .4779 (.0171) & .5098 (.0289) & .5863 (.0227) & .5916 (.0371) & .7597 (.007) \\
%   Eve & .5408 (.036)  & .4905 (.0183) & .5408 (.0186) & .6125 (.0229) & .6189 (.0287) & .7862 (.0058) \\
%   Naomi & .5250 (.0269) & .4922 (.0205) & .3860 (.028) & .6019 (.0218) & .6021 (.0322) & .7672 (.0071) \\
%   Nina & .5412 (.0304) & .5032 (.0213) & .3950 (.0223) & .6308 (.0277) & .6474 (.0331) & .8089 (.0089) \\
%   Peter & .5359 (.036) & .5092 (.0216) & .5359 (.0188) & .6250 (.0252) & .6315 (.0263) & .7974 (.0082) \\
%   \hline
% \end{tabular}
% \end{table}

%%% Whole results
% \begin{table}[ht]
% \small
% \centering
% \caption{10 fold cross-validation accuracy of the connectionist model on each
%   child corpus with the standard labelling are summarized.  The training
%   phase of each corpus is stopped after 50K word patterns are presented.  Standard
%   errors are reported in parentheses.
% }

% \begin{tabular}{|c|c@{ }|c@{ }|c@{ }|c@{ }|c@{ }|c@{ }|c@{ }|}
%   \hline
%   & \multicolumn{4}{c|}{Frames} & \multicolumn{2}{c|}{\specialcell{Number of \\ Substitute Words}}\\
%   \hline
%   Child & aX & Xb & aXb & aX + Xb & 1 & 16\\
%   \hline
%   Anne & .6165 (.0119)  & .6150 (.0119) & .5251 (.0189) & .7545 (.0147) & .7321 (.0081) & {\bf .8273 (.0087)} \\
%   Aran & .5924 (.0193) & .5583 (.0193) & .4834 (.0131) & .7164 (.0151) & .7081 (.0074) & {\bf .8136 (.0096)} \\
%   Eve & .6337 (.0139)  & .5850 (.0116) & .5425 (.0198) & .7605 (.0104) & .7435 (.0208) & {\bf .8378 (.0199)} \\
%   Naomi & .6183 (.0165) & .5908 (.0254) & .5353 (.0234) & .7438 (.0156) & .7146 (.0136) & {\bf .8165 (.0147)} \\
%   Nina & .6353 (.0195) & .6097 (.0049) & .5556 (.0106) & .7745 (.0199) & .7527 (.0059) & {\bf .8494 (.0073)} \\
%   Peter & .6056 (.0205) & .6232 (.0205) & .5604 (.0115) & .7630 (.005) & .7398 (.0120) & {\bf .8437 (.0061)} \\
%   \hline
% \end{tabular}
% \end{table}
% \begin{table}[h]
% \label{tab:myfirsttable}
% \end{table}




\section{Experiment 2}
Number of subsitutes.  We need to show 16 is better than 1 but 16+ is same
\subsection{Input Corpora}
\subsection{Method}
\subsection{Results}

\section{Experiment 3}
N-gram order 2,3

\section{Experiment 4}
What happens when we change the data size?\\
What happens when we change the vocabulary threshold?\\

\subsection{Input Corpora}
\subsection{Method}
\subsection{Results}

\section{Experiment 5}
Left/rigth context substitute
\subsection{Input Corpora}
\subsection{Method}
\subsection{Results}

\section{Experiment 6}
Other languages that we have in CHILDES

\section{Experiment 7}
What happens if some of the words are given (semi-supervised setting)

\subsection{Input Corpora}
\subsection{Method}
\subsection{Results}



